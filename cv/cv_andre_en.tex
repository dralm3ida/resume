%%%%%%%%%%%%%%%%%%%%%%%%%%%%%%%%%%%%%%%%%
% "ModernCV" CV and Cover Letter
% LaTeX Template
% Version 1.3 (29/10/16)
%
% This template has been downloaded from:
% http://www.LaTeXTemplates.com
%
% Original author:
% Xavier Danaux (xdanaux@gmail.com) with modifications by:
% Vel (vel@latextemplates.com)
%
% License:
% CC BY-NC-SA 3.0 (http://creativecommons.org/licenses/by-nc-sa/3.0/)
%
% Important note:
% This template requires the moderncv.cls and .sty files to be in the same 
% directory as this .tex file. These files provide the resume style and themes 
% used for structuring the document.
%
%%%%%%%%%%%%%%%%%%%%%%%%%%%%%%%%%%%%%%%%%

%----------------------------------------------------------------------------------------
%	PACKAGES AND OTHER DOCUMENT CONFIGURATIONS
%----------------------------------------------------------------------------------------

\documentclass[11pt,a4paper,sans]{moderncvext_en} % Font sizes: 10, 11, or 12; paper sizes: a4paper, letterpaper, a5paper, legalpaper, executivepaper or landscape; font families: sans or roman

\moderncvstyle{casual} % CV theme - options include: 'casual' (default), 'classic', 'oldstyle' and 'banking'
\moderncvcolor{blue} % CV color - options include: 'blue' (default), 'orange', 'green', 'red', 'purple', 'grey' and 'black'

\usepackage{blindtext}
\usepackage{lipsum} % Used for inserting dummy 'Lorem ipsum' text into the template

\usepackage[scale=0.75]{geometry} % Reduce document margins
%\setlength{\hintscolumnwidth}{3cm} % Uncomment to change the width of the dates column
%\setlength{\makecvtitlenamewidth}{10cm} % For the 'classic' style, uncomment to adjust the width of the space allocated to your name

\usepackage{fontawesome}
\usepackage[utf8]{inputenc}
\usepackage[english,portuguese]{babel}
\selectlanguage{portuguese}


\setcounter{tocdepth}{2} %subsections are added to the TOC
\setcounter{secnumdepth}{4} %subsubsections are numbered

%--------------------------------------------





%------------------------------------------------






%----------------------------------------------------------------------------------------
%	NAME AND CONTACT INFORMATION SECTION
%----------------------------------------------------------------------------------------

\firstname{André} % Your first name
\familyname{Rodrigues Almeida} % Your last name

% All information in this block is optional, comment out any lines you don't need
\title{Curriculum Vitae}
\address{Aveiro}{Portugal}
\mobile{969136952}
%\phone{(000) 111 1112}
%\fax{(000) 111 1113}
\email{andre.r.alm3ida@gmail.com / andre.rodrigues@ua.pt}
\homepage{https://www.linkedin.com/in/andre-r-almeida/}{https://www.linkedin.com/in/andre-r-almeida/}
%\homepage{staff.org.edu/~jsmith}{staff.org.edu/$\sim$jsmith} % The first argument is the url for the clickable link, the second argument is the url displayed in the template - this allows special characters to be displayed such as the tilde in this example
%\extrainfo{additional information}
\photo[70pt][0.4pt]{pictures/Foto} % The first bracket is the picture height, the second is the thickness of the frame around the picture (0pt for no frame)
%\quote{"A witty and playful quotation" - John Smith}

%----------------------------------------------------------------------------------------

\begin{document}

%----------------------------------------------------------------------------------------
%	COVER LETTER
%----------------------------------------------------------------------------------------

% To remove the cover letter, comment out this entire block

%\clearpage

%\recipient{HR Department}{Corporation\\123 Pleasant Lane\\12345 City, State} % Letter recipient
%\date{\today} % Letter date
%\opening{Dear Sir or Madam,} % Opening greeting
%\closing{Sincerely yours,} % Closing phrase
%\enclosure[Attached]{curriculum vit\ae{}} % List of enclosed documents

%\makelettertitle % Print letter title

%\lipsum[1-2] % Dummy text
%\lipsum[4] % Dummy text

%\makeletterclosing % Print letter signature

%\newpage

%----------------------------------------------------------------------------------------
%	CURRICULUM VITAE
%----------------------------------------------------------------------------------------

\makecvtitle % Print the CV title

%----------------------------------------------------------------------------------------
%	INTRO SECTION
%----------------------------------------------------------------------------------------

\section{Introduction}

% Intro, career objective and potential: q.b.
\subsection{About me}
\cvitem{}{Professional with several years of experience in software development in Frontend and Backend. I am interested in deepening my knowledge in software systems conception and design, both in arquitectural and implementation levels.}

% Compact personal info
\subsection{Personal Info}

\cvitem{Nationality}{\mynationality}
\cvitem{Residence}{\myaddress}
\cvitem{\faPhone}{969136952}
\cvitem{\faEnvelope}{andre.r.alm3ida@gmail.com / andre.rodrigues@ua.pt}
\cvitem{\faLinkedinSquare}{\url{https://www.linkedin.com/in/andre-r-almeida/}}
\cvitem{\faGithub}{\url{https://github.com/dralm3ida}}


%----------------------------------------------------------------------------------------
%	TECHNICAL SKILLS SECTION
%----------------------------------------------------------------------------------------

\section{Skills}

\subsection{General}

\cvitem{Platforms}{Windows / Linux}
\cvitem{Programming languages}{\textsc{java}, Js, C, \cpp, \textsc{php}, \textsc{Python}, Shell Script, \textbf{markup}(HTML\faHtml5, XML) \textbf{style sheet}(CSS)}
\cvitem{Frameworks \& DB}{SpringBoot, Hibernate, SQLite, PostgreSQL, AngularJS, FastAPI}
\cvitem{Tools}{Git, SVN, Ms-Office, LaTeX, VSCode, Intellij IDEA, JIRA)}

%----------------------------------------------------------------------------------------
%	WORK EXPERIENCE SECTION
%----------------------------------------------------------------------------------------

\section{Professional Experience}

\subsection{Altice Labs - Aveiro}
\cventry{06-2021--Current}{Fullstack Software Engineer}{}{}{}{
\textbf{Projects:}
\begin{itemize}
\item \textbf{VNOs:} (Virtual Network Operator) - Ongoing development of the back-end of two applications within the scope of network virtualization for operators. \emph{VNO Manager}:application that intends to abstract the physical network of the InP (Infrastructure Provider) into a virtual network so that the VNOs can share the same physical resources independently. \emph{VNO Portal}: each operator's network management application, so that it can manage its operations. \textbf{Tecnologias:} JAVA e Spring Boot.
\item \textbf{CPEs} Project monitoring and management.
\newline{}
\end{itemize} 
}

%------------------------------------------------

\cventry{06-2015--06-2021}{Front End Software Engineer}{}{}{}{
\textbf{Projetos:}
\begin{itemize}
\item \textbf{CPEs:} (Costumer Provided Equipment) - Responsible for the design and development of the \emph{WebGUI} client application for the various GPON (FiberGateway), DOCSIS, Entry Level terminal equipment developed at Altice Labs; Responsible for the technological follow-up of the product, namely in Smart Wi-Fi technology and its integration in the \emph{WebGUI}. \textbf{Tecnologias:} AngularJS;
\item \textbf{Smart Wi-Fi Extendor:} Responsible for the design and development of the client's WebGUI application. \textbf{Technologies:} AngularJS;
\item \textbf{AGORA:} involved in the development of the Front end resource manager application for managing network equipment. \textbf{Technologies:} AngularJS;
\item \textbf{WDG:} (Web Develpment Group) - participation in an informal company group with members from various departments with the aim of sharing knowledge of new technologies and methodologies in the context of web development.
\item \textbf{OLT:} Product client web application development \emph{WebTi}. \textbf{Technologies:} AngularJS;
\item \textbf{uMSAN48V:} - Involved in WebTi client web application development. \textbf{Technologies:} HTML, JS, CSS;
\newline{}
\end{itemize} 
}

%------------------------------------------------

\cventry{08-2013--11/2016}{Embedded Systems Software Engineer}{}{}{}{
\textbf{Projetos:}
\begin{itemize}
\item \textbf{uMSAN48V:} ("triple play" terminal equipment for reuse of existing copper network) - Development of local management applications \emph{Manager} and \emph{CGI XML Parser}. \textbf{Technologies:} C, XML.
\item \textbf{OLT:} (Optical Line Terminator) - Occasionally involved in the development of the product's core \emph{Manager} and \emph{CGIs} applications. \textbf{Technologies:} C, XML.
\item \textbf{RFO:} (Radio Frequency Overlay) - Development of CLI application for equipment management. \textbf{Technologies:} C.
\newline{}
\end{itemize} 
}

%----------------------------------------------------------------------------------------
%	ACADEMIC FORMATION SECTION
%----------------------------------------------------------------------------------------

\section{Academic education}

\cventry{2020--2020}{microMBA Postgraduate}{Aveiro University}{}{}{
Development of skills in the area of business management, more specifically in marketing, organizational behavior, strategy, operations and finance.
}

\cventry{2007--2013}{Higher education}{Aveiro University}{\emph{master's} degree}{}{Integrated Master of Electronic Engineering and Telecommunications.
\newline
\newline
\textbf{Masters dissertation:}
\begin{itemize}
\item \textbf{Name:} Automotive Ethernet Resource Reservation; 
\item \textbf{Description:} Development of simulations of the resource reservation mechanism SRP (Stream Reservation Protocol) in OMNeT++. Models and simulations built in \cpp over the INET framework in automotive Ethernet network scenarios.
\end{itemize}
}

%----------------------------------------------------------------------------------------
%	EVENTS SECTION
%----------------------------------------------------------------------------------------

\section{Events}

%\cventry{7, 8 e 9 de Setembro de 2018}{\href{https:sunsethackathon.com/}{Sunset Hackathon 2018}}{realizado pelo %\href{https://hardwarecity.org/}{Hardware city}}{\textbf{Tema:} %Domótica; PLC´s; Sensores; IoT;}{}{\textbf{Desafio:} 
%}

\cventry{7, 8 and 9 of September 2018}{\href{https://sunsethackathon.com}{Sunset Hackathon 2018}}{organized by \href{https://hardwarecity.org/}{Hardware city}}{\textbf{Themes:} Home automation; PLCs; Sensors; IoT}{}{\textbf{Challenge:} Participation in a team of 5 elements where it was proposed the development of a light control system (prototype) in a hotel room controlled by voice commands and through automatic detection of occupancy of space and movement in the room. Link: \url{https://github.com/r0drig/roomControlLights}
\newline
}

\cventry{21 and 22 of February 2018}{Hackathon Altice Labs}{organized by Future Labs of Altice Labs}{\textbf{Theme:} solve a specific problem related to Altice Labs Campus, in order to make it more practical, sustainable and dynamic. Contribute to new smart living products and services.}{}{
Participation in a team of 3 elements, having reached 2nd place -- \href{http://www.alticelabs.com/pt/338-hackathon-altice-labs-2018-hal2018.html}{HAL2018}
\newline
}

%----------------------------------------------------------------------------------------
%	COMPLEMENTARY FORMATION SECTION
%----------------------------------------------------------------------------------------

\section{Additional formation}

\cventry{04-07-2019}{Workshop}{Kubernetes}{organized by Altice Labs}{}{Platform to orchestrate services in containers and automate its deployment.}

\cventry{30-05-2019}{Techdays}{100º Tech Day Training Session about Docker technology}{promoted by Altice Labs}{}{How to deploy applications in micro-service based architectures.}

\cventry{08-09-2018}{Workshop}{Raspberry Pi}{promoted by Hardware city}{}{Introduction to raspberry pi and GPIOs.}

\cventry{15/06/2018}{Workshop}{UNITY}{organized by Altice Labs}{}{Introduction to the UNITY tool (interactive application creation platform).}

\cventry{07/02/2018}{Workshop}{LUNA}{organized by Altice Labs}{}{How to build interactive television applications using the LUNA platform.}

\cventry{2016}{Free Japanese Course}{held at the University of Aveiro}{with the duration of 48 hours}{}{}

\cventry{2015--2016}{Online course in “Full Stack Web Development”}{from the University of Hong Kong}{sponsored by the "Coursera" education platform.}{}{
\begin{itemize}
\item \href{https://www.coursera.org/account/accomplishments/certificate/AY6KCDJD9EE6}{HTML, CSS and JavaScrip}
\item \href{https://www.coursera.org/account/accomplishments/certificate/BDC6WHU2GNEW}{Front-End Web UI Frameworks and Tools}
\item \href{https://www.coursera.org/account/accomplishments/certificate/8VVLDFKSPXYD}{Front-End JavaScript Frameworks: AngularJS}
\item \href{https://www.coursera.org/account/accomplishments/certificate/FNYJFTJPTRXS}{Multiplatform Mobile App Development with Web
Technologies}
\end{itemize}
}

\cventry{2015}{Online course in “Software Security”}{from the Universidade of Maryland}{sponsored by the "Coursera" education platform.}{}{
\begin{itemize}
\item \href{https://www.coursera.org/account/accomplishments/certificate/GHDH797YBT}{Software Security}
\end{itemize}
}

\cventry{2012}{Workshop}{Automation}{Held in Bresimar company in Cacia}{}{}

\cventry{2012}{Free German Course}{held at the University of Aveiro}{with the duration of 60 hours}{corresponding to the A1.1 level of QECR}{}

\cventry{2007}{English course}{International House school}{Viseu}{}{FCE -- “First Certificate in English” Achievement from the University of Cambridge, corresponding to the B2 level of QECR.}

%----------------------------------------------------------------------------------------
%	AWARDS SECTION
%----------------------------------------------------------------------------------------

%\section{Awards}

%\cvitem{2011}{School of Business Postgraduate Scholarship}
%\cvitem{2010}{Top Achiever Award -- Commerce}

%----------------------------------------------------------------------------------------
%	COMMUNICATION SKILLS SECTION
%----------------------------------------------------------------------------------------

%\section{Communication Skills}

%\cvitem{2010}{Oral Presentation at the California Business Conference}
%\cvitem{2009}{Poster at the Annual Business Conference in Oregon}

%----------------------------------------------------------------------------------------
%	LANGUAGES SECTION
%----------------------------------------------------------------------------------------

%\section{Languages}

%\cvitemwithcomment{English}{Mothertongue}{}
%\cvitemwithcomment{Spanish}{Intermediate}{Conversationally fluent}
%\cvitemwithcomment{Dutch}{Basic}{Basic words and phrases only}

%----------------------------------------------------------------------------------------
%	INTERESTS SECTION
%----------------------------------------------------------------------------------------

%\section{Interests}

%\renewcommand{\listitemsymbol}{-~} % Changes the symbol used for lists

%\cvlistdoubleitem{Piano}{Chess}
%\cvlistdoubleitem{Cooking}{Dancing}
%\cvlistitem{Running}

%----------------------------------------------------------------------------------------

\end{document}