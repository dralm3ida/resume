%%%%%%%%%%%%%%%%%%%%%%%%%%%%%%%%%%%%%%%%%
% "ModernCV" CV and Cover Letter
% LaTeX Template
% Version 1.3 (29/10/16)
%
% This template has been downloaded from:
% http://www.LaTeXTemplates.com
%
% Original author:
% Xavier Danaux (xdanaux@gmail.com) with modifications by:
% Vel (vel@latextemplates.com)
%
% License:
% CC BY-NC-SA 3.0 (http://creativecommons.org/licenses/by-nc-sa/3.0/)
%
% Important note:
% This template requires the moderncv.cls and .sty files to be in the same 
% directory as this .tex file. These files provide the resume style and themes 
% used for structuring the document.
%
%%%%%%%%%%%%%%%%%%%%%%%%%%%%%%%%%%%%%%%%%

%----------------------------------------------------------------------------------------
%	PACKAGES AND OTHER DOCUMENT CONFIGURATIONS
%----------------------------------------------------------------------------------------

\documentclass[11pt,a4paper,sans]{moderncvext} % Font sizes: 10, 11, or 12; paper sizes: a4paper, letterpaper, a5paper, legalpaper, executivepaper or landscape; font families: sans or roman

\moderncvstyle{casual} % CV theme - options include: 'casual' (default), 'classic', 'oldstyle' and 'banking'
\moderncvcolor{blue} % CV color - options include: 'blue' (default), 'orange', 'green', 'red', 'purple', 'grey' and 'black'

\usepackage{blindtext}
\usepackage{lipsum} % Used for inserting dummy 'Lorem ipsum' text into the template

\usepackage[scale=0.75]{geometry} % Reduce document margins
%\setlength{\hintscolumnwidth}{3cm} % Uncomment to change the width of the dates column
%\setlength{\makecvtitlenamewidth}{10cm} % For the 'classic' style, uncomment to adjust the width of the space allocated to your name

\usepackage{fontawesome}
\usepackage[utf8]{inputenc}
\usepackage[english,portuguese]{babel}
\selectlanguage{portuguese}


\setcounter{tocdepth}{2} %subsections are added to the TOC
\setcounter{secnumdepth}{4} %subsubsections are numbered

%--------------------------------------------





%------------------------------------------------






%----------------------------------------------------------------------------------------
%	NAME AND CONTACT INFORMATION SECTION
%----------------------------------------------------------------------------------------

\firstname{André} % Your first name
\familyname{Rodrigues Almeida} % Your last name

% All information in this block is optional, comment out any lines you don't need
\title{Curriculum Vitae}
\address{Aveiro}{Portugal}
\mobile{969136952}
%\phone{(000) 111 1112}
%\fax{(000) 111 1113}
\email{andralmeida0@gmail.com / andre.rodrigues@ua.pt}
\homepage{https://www.linkedin.com/in/andre-r-almeida/}{https://www.linkedin.com/in/andre-r-almeida/}
%\homepage{staff.org.edu/~jsmith}{staff.org.edu/$\sim$jsmith} % The first argument is the url for the clickable link, the second argument is the url displayed in the template - this allows special characters to be displayed such as the tilde in this example
%\extrainfo{additional information}
\photo[70pt][0.4pt]{pictures/Foto} % The first bracket is the picture height, the second is the thickness of the frame around the picture (0pt for no frame)
%\quote{"A witty and playful quotation" - John Smith}

%----------------------------------------------------------------------------------------

\begin{document}

%----------------------------------------------------------------------------------------
%	COVER LETTER
%----------------------------------------------------------------------------------------

% To remove the cover letter, comment out this entire block

%\clearpage

%\recipient{HR Department}{Corporation\\123 Pleasant Lane\\12345 City, State} % Letter recipient
%\date{\today} % Letter date
%\opening{Dear Sir or Madam,} % Opening greeting
%\closing{Sincerely yours,} % Closing phrase
%\enclosure[Attached]{curriculum vit\ae{}} % List of enclosed documents

%\makelettertitle % Print letter title

%\lipsum[1-2] % Dummy text
%\lipsum[4] % Dummy text

%\makeletterclosing % Print letter signature

%\newpage

%----------------------------------------------------------------------------------------
%	CURRICULUM VITAE
%----------------------------------------------------------------------------------------

\makecvtitle % Print the CV title

%----------------------------------------------------------------------------------------
%	INTRO SECTION
%----------------------------------------------------------------------------------------

\section{Introdução}

% Intro, career objective and potential: q.b.
\subsection{Sobre mim}
\cvitem{}{Curso obtido em Engenharia de Eletrónica e Telecomunicações, a trabalhar numa organização tecnológica da área das telecomunicações. Com interesse profissional nas áreas de desenvolvimento de software de sistemas e aplicações web (front-end).}

% Compact personal info
\subsection{Informação pessoal}


\cvitem{Data de nascimento}{\mybirthday}
\cvitem{Nacionalidade}{\mynationality}
\cvitem{Residência}{\myaddress}
\cvitem{\faPhone}{969136952}
\cvitem{\faEnvelope}{andralmeida0@gmail.com / andre.rodrigues@ua.pt}
\cvitem{\faLinkedinSquare}{\url{https://www.linkedin.com/in/andre-r-almeida/}}
\cvitem{\faGithub}{\url{https://github.com/r0drig}}



%----------------------------------------------------------------------------------------
%	TECHNICAL SKILLS SECTION
%----------------------------------------------------------------------------------------

\section{Competências}

\subsection{Geral}

\cvitem{Platarmas}{Windows / Linux}
\cvitemwithcomment{Lingua}{Inglês}{Conversação fluente e competência em escrita e compreensão }
%\cvitemwithcomment{Spanish}{Intermediate}{Conversationally fluent}
%\cvitemwithcomment{Dutch}{Basic}{Basic words and phrases only}

\subsection{Linguagens de Programação e frameworks}

\cvitem{Competente em}{C, Javascript, php, AngularJS, \textbf{markup}(HTML\faHtml5, XML) \textbf{style sheet}(CSS)}
\cvitem{Conhecimentos básicos em}{\textsc{java}, \cpp, MatLab}
%\cvitem{Basic}{\textsc{java}, Adobe Illustrator}
%\cvitem{Intermediate}{\textsc{python}, \textsc{html}, \LaTeX, OpenOffice, Linux, Microsoft Windows}
%\cvitem{Advanced}{Computer Hardware and Support}

\subsection{Ferramentas computacionais}

\cvitem{}{Git, SVN, Ms-Office, LaTeX, Visual Studio Code, Eclipse, Intellij IDEA, Atlassian (Jira, wiki, Fisheye)}

%----------------------------------------------------------------------------------------
%	WORK EXPERIENCE SECTION
%----------------------------------------------------------------------------------------

\section{Experiência Profissional}

\subsection{Altice Labs}

\cventry{16/08/2013--A decorrer}{Engenheiro de software}{\textsc{Altice Labs}}{Aveiro}{}{ 
\begin{itemize}
\item \textbf{Empregador} - Withus-Inovação e Tecnologia, Lda, Rua Dr. Mário Sacramento, Edificio Colombo I, 177, 1.o Q, 3810-106.
\item \textbf{Descrição} - De momento encontro-me a trabalhar como prestador de serviços no departamento de Desenvolvimentos de Sistemas de Redes da Altice Labs - R. Eng. José Ferreira Pinto Basto, Aveiro, 3810-106 PORTUGAL. Desempenho funções de desenvolvedor de software no âmbito do grupo da Gestão Local de equipamentos de rede.
\end{itemize}
}

\subsection{Projectos profissionais principais}

\cventry{05-2019--a decorrer}{SmartMesh Wi-Fi}{}{}{}{Serviço end-to-end para disponibilização de rede Smart mesh Wi-Fi em casa dos clientes.
\textbf{Envolvimento:} Desenvolvimento da aplicação Web de gestão local de equipamentos extenders de rede, quando estes funcionam como route AP (cenário SmartMesh Standalone).
} 

\cventry{01-2017--a decorrer}{Produtos CPEs (Costumer Provider Equipments)}{}{}{}{Produtos de cliente de fim de rede nas seguintes tecnologias:
\begin{itemize}
\item \textbf{FGW (FiberGateway):} Gateway de fibra óptica em cenários FTTH e B2B;
\item \textbf{DOCSIS} Gateway de cabo coaxial.
\newline{}
\end{itemize} 
\textbf{Envolvimento:} Desenvolvimento da aplicação Web de gestão local destes equipamentos para os seguintes operadores: MEO, SFR.
\begin{itemize}
\item Desenvolvimento da aplicação feita em AngularJS e em integração continua com a equipa de desenvolvimento do FW do equipamento e em colaboração com a equipa de usabilidade no desenho da mesma;
\item Orientação de novo colaborador no desenvolvimento e suporte da mesma.
\end{itemize} 
} 
%\begin{itemize}
%\item Learned how to make amazing coffee
%\item Finally determined the reason for \textsc{PC LOAD LETTER}:
%\begin{itemize}
%\item Paper jam
%\item Software issues:
%\begin{itemize}
%\item Word not sending the correct data to printer
%\item Windows trying to print in letter format
%\end{itemize}
%\item Coffee spilled inside printer
%\end{itemize}
%\item Broke the office record for number of kitten pictures in cubicle
%\end{itemize}}

%------------------------------------------------

\cventry{11-2014--a decorrer}{OLT}{}{}{}{Equipamentos de core da rede com tecnologia Ethernet, GPON e NGPON2, que permite gerir equipamentos terminais (ONUs).
\newline{}\newline{}
\textbf{Envolvimento:}
\begin{itemize}
\item Envolvido num grupo de 4 pessoas no desenvolvimento da interface gráfica do produto em AngularJS (descontinuado).
\item Envolvimento ocasional ao nível das da aplicação de core (manager) e CGIs do produto.
\end{itemize} 
}

%------------------------------------------------

\cventry{10-2013--11-2017}{uMSAN48V}{}{}{}{Equipamento terminal com tecnologia GPON e Ethernet de ‘uplink’ que permite a reutilização das redes de cobre já existentes para disponibilizar serviços de ‘triple play’ (voz, dados e vídeo).
\newline{}\newline{}
\textbf{Envolvimento:} Envolvido num grupo de 3 pessoas nas seguintes aplicações:
\begin{itemize}
\item Aplicação de gestão de equipamento (backend e regras de negócio); \item Aplicação CGI de parser XML;
\item Aplicação cliente web (WebTi).
\end{itemize}
Desenvolvimento em C e javascript.
\newline
}

%------------------------------------------------

\cventry{08-2013--11-2014}{RFO – Radio Frequency Overlay.)}{}{}{}{Equipamento de transporte de sinais de TV analógica sobre fibra ótica.
\newline{}\newline{}
\textbf{Envolvimento:} Desenvolvimento de aplicação CLI em C para interação (consulta de estados e configurações) do equipamento.
\newline
}

%------------------------------------------------

\subsection{Projectos profissionais secundários}

\cventry{06-02-2019--a decorrer}{WDG (Web Development Group)}{}{}{}{Participação num grupo informal ao nível da empresa, com reuniões mensais, com o objectivo de partilhar e dar a conhecer novas tecnologias e metedologias no desenvolvimento de software de outros departamentos.
\newline
}

\cventry{05-2018--10-2018}{Trakiii}{realização do projecto no âmbito do \href{http://www.alticelabs.com/pt/338-hackathon-altice-labs-2018-hal2018.html}{HAL2018}}{}{}{\textbf{Objectivo:} Realização de um sistema que permita detectar através de uma câmara e estimar o tamanho médio de uma fila de pessoas, bem como o tempo médio de espera para o atendimento. Cenário utilizado foi a cantina do campus da Altice Labs.
\newline{}\newline{}
\textbf{Envolvimento:} Envolvido num grupo de 4 pessoas na idealização e desenvolvimento do algoritmo de processamento de imagem e calculo da estimativa de ocupação de fila. Desenvolvimento do core do sistema em \cpp, com recurso à biblioteca de processamento de imagem OpenCV.
}

%------------------------------------------------
%\subsection{Miscellaneous}

%\cventry{2010--2011}{}{}{}{}{Spent some time finding myself. This was a courageous endeavour that didn't have a job title. It was quite important to my overall development though so I'm adding it to my CV. Also it explains the gap in my otherwise stellar CV.}

%\cventry{2009--2010}{Computer Repair Specialist}{Buy More}{Burbank}{}{Worked in the Nerd Herd and helped to solve computer problems. Allowed me to become expert in all forms of martial arts and weaponry.}

%----------------------------------------------------------------------------------------
%	ACADEMIC FORMATION SECTION
%----------------------------------------------------------------------------------------

\section{Formação Académica}

\cventry{2007--2013}{Ensino superior}{Universidade de Aveiro}{grau de \emph{Mestre}}{}{Mestrado Integrado de Engenharia Eletrónica e Telecomunicações.
\newline
\newline
\textbf{Dissertação de Mestrado:}
\begin{itemize}
\item \textbf{Nome:} Reserva de recursos para Automotive Ethernet; 
\item \textbf{Descrição:} Desenvolvimento de simulações em
OMNeT++ do mecanismo de reserva de recursos SRP –
Stream Reservation Protocol. Modelos construídos em C++ sobre a framework INET e execução de simulações no cenário de redes Automotive Ethernet.
\end{itemize}
} 

\cventry{2004--2007}{Ensino secundário}{Escola Secundária Alves Martins}{Viseu}{}{Ciências e Tecnologias}


%\cventry{2011--2012}{Masters of Commerce}{The University of California}{Berkeley}{\textit{GPA -- 8.0}}{First Class Honours}  % Arguments not required can be left empty
%\cventry{2007--2010}{Bachelor of Business Studies}{The University of California}{Berkeley}{\textit{GPA -- 7.5}}{Specialized in Commerce}

%\section{Masters Thesis}

%\cvitem{Title}{\emph{Money Is The Root Of All Evil -- Or Is It?}}
%\cvitem{Supervisors}{Professor James Smith \& Associate Professor Jane Smith}
%\cvitem{Description}{This thesis explored the idea that money has been the cause of untold anguish and suffering in the world. I found that it has, in fact, not.}

%----------------------------------------------------------------------------------------
%	EVENTS SECTION
%----------------------------------------------------------------------------------------

\section{Eventos}

%\cventry{7, 8 e 9 de Setembro de 2018}{\href{https:sunsethackathon.com/}{Sunset Hackathon 2018}}{realizado pelo %\href{https://hardwarecity.org/}{Hardware city}}{\textbf{Tema:} %Domótica; PLC´s; Sensores; IoT;}{}{\textbf{Desafio:} 
%}

\cventry{7, 8 e 9 de Setembro de 2018}{\href{https://sunsethackathon.com}{Sunset Hackathon 2018}}{realizado pelo \href{https://hardwarecity.org/}{Hardware city}}{\textbf{Tema:} Domótica; PLCs; Sensores; IoT}{}{\textbf{Desafio:} Participação numa equipa de 5 elementos onde foi proposto o desenvolvimento de um sistema de controlo de luzes (protótipo) de um quarto de hotel por comandos de voz e por deteção automática de ocupação de espaço e movimento. Link: \url{https://github.com/r0drig/roomControlLights}
}

\cventry{21 e 22 de Fevereiro de 2018}{Hackathon Altice Labs}{realizado pelo grupo Future Labs da Altice Labs}{\textbf{Mote:} resolver um problema específico do Campus da Altice Labs, de forma a torná-lo mais prático, sustentável e dinâmico. Contribuir para novos produtos e serviços Smart Living}{}{
Participação do evento numa equipa de 3 elementos tendo alcançado o 2º lugar -- \href{http://www.alticelabs.com/pt/338-hackathon-altice-labs-2018-hal2018.html}{HAL2018}
}

%----------------------------------------------------------------------------------------
%	COMPLEMENTARY FORMATION SECTION
%----------------------------------------------------------------------------------------

\section{Formação Complementar}

\cventry{04-07-2019}{Workshop}{Kubernetes}{promovido pela Altice Labs}{}{Plataforma de orquestração de serviços em containers e da automatização do deployment dos mesmos.}

\cventry{30-05-2019}{Techdays}{100º Tech Day Sessão de Formação sobre Docker}{promovido pela Altice Labs}{}{Como fazer deploy de aplicações em aquitecturas baseadas em micro-serviços.}

\cventry{08-09-2018}{Workshop}{Raspberry Pi}{promovido pelo Hardware city}{}{Introdução, Visão por Computador e GPIOs.}

\cventry{15/06/2018}{Workshop}{UNITY}{realizado na Altice Labs}{}{Introdução à ferramenta  UNITY (plataforma para criação de aplicações interativas)}

\cventry{07/02/2018}{Workshop}{LUNA}{realizado na Altice Labs}{}{Como construir aplicações interativas para televisão, usando a plataforma LUNA}

\cventry{2016}{Curso livre de “Japonês”}{realizado na Universidade de Aveiro}{com duração de 48 horas}{}{}

\cventry{2015--2016}{Curso online de “Full Stack Web Development”}{da Universidade de Hong Kong}{patrocinado pela plataforma de educação “Coursera”}{}{
\begin{itemize}
\item \href{https://www.coursera.org/account/accomplishments/certificate/AY6KCDJD9EE6}{HTML, CSS and JavaScrip}
\item \href{https://www.coursera.org/account/accomplishments/certificate/BDC6WHU2GNEW}{Front-End Web UI Frameworks and Tools}
\item \href{https://www.coursera.org/account/accomplishments/certificate/8VVLDFKSPXYD}{Front-End JavaScript Frameworks: AngularJS}
\item \href{https://www.coursera.org/account/accomplishments/certificate/FNYJFTJPTRXS}{Multiplatform Mobile App Development with Web
Technologies}
\end{itemize}
}

\cventry{2015}{Curso online de “Software Security”}{da Universidade de Maryland}{patrocinado pela plataforma de educação “Coursera”}{}{
\begin{itemize}
\item \href{https://www.coursera.org/account/accomplishments/certificate/GHDH797YBT}{Software Security}
\end{itemize}
}

\cventry{2012}{Workshop}{Automação}{Empresa Bresimar em Cacia}{}{}

\cventry{2012}{Curso livre de “Alemão”}{realizado na Universidade de Aveiro}{com duração de 60 horas}{correspondente ao nível A1.1 do QECR}{}

\cventry{2007}{Curso de “Inglês”}{Escola International House}{Viseu}{}{Realização do exame da Universidade de Cambridge FCE -- “First Certificate in English” correspondente ao nível B2 do QECR.}

%----------------------------------------------------------------------------------------
%	AWARDS SECTION
%----------------------------------------------------------------------------------------

%\section{Awards}

%\cvitem{2011}{School of Business Postgraduate Scholarship}
%\cvitem{2010}{Top Achiever Award -- Commerce}

%----------------------------------------------------------------------------------------
%	COMMUNICATION SKILLS SECTION
%----------------------------------------------------------------------------------------

%\section{Communication Skills}

%\cvitem{2010}{Oral Presentation at the California Business Conference}
%\cvitem{2009}{Poster at the Annual Business Conference in Oregon}

%----------------------------------------------------------------------------------------
%	LANGUAGES SECTION
%----------------------------------------------------------------------------------------

%\section{Languages}

%\cvitemwithcomment{English}{Mothertongue}{}
%\cvitemwithcomment{Spanish}{Intermediate}{Conversationally fluent}
%\cvitemwithcomment{Dutch}{Basic}{Basic words and phrases only}

%----------------------------------------------------------------------------------------
%	INTERESTS SECTION
%----------------------------------------------------------------------------------------

%\section{Interests}

%\renewcommand{\listitemsymbol}{-~} % Changes the symbol used for lists

%\cvlistdoubleitem{Piano}{Chess}
%\cvlistdoubleitem{Cooking}{Dancing}
%\cvlistitem{Running}

%----------------------------------------------------------------------------------------

\end{document}